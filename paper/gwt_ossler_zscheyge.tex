\documentclass[12pt, a4paper, ngerman]{article}
\usepackage{babel}
\usepackage[utf8]{inputenc}

\usepackage[numbers]{natbib}

%\usepackage{ucs}
\usepackage{amsmath}
\usepackage{amsfonts}
\usepackage{amssymb}

\usepackage[T1]{fontenc}
\usepackage[bitstream-charter]{mathdesign}
\usepackage[pdftex]{graphicx}
\usepackage[usenames, dvipsnames]{color}

\usepackage{vmargin}
\usepackage{float}
\usepackage[font=small, format=plain, labelfont=bf]{caption}
\usepackage{url}
\usepackage{verbatim}
\usepackage{listings}
\usepackage[colorlinks]{hyperref}

\usepackage{fancyhdr}
%\usepackage[bottom]{footmisc}

%% Define a new 'leo' style for the package that will use a smaller font.
\makeatletter
\def\url@leostyle{%
  \@ifundefined{selectfont}{\def\UrlFont{\sf}}{\def\UrlFont{\small\ttfamily}}}
\makeatother
%% Now actually use the newly defined style.
\urlstyle{leo}

\setmarginsrb{2.5cm}{1.5cm}{2.5cm}{1.5cm}%
			 {7mm}{1.2cm}{4mm}{1.5cm}

\author{Philipp Ossler, Oliver Zscheyge}
\title{Anwendungsentwicklung mit dem Google Web Toolkit}

\setlength{\parindent}{0pt}

% Document begin
% ==============
\begin{document}
\pagestyle{plain}

ref:learnmore
http://code.google.com/webtoolkit/learnmore-sdk.html

Entry Point (http://code.google.com/webtoolkit/doc/latest/DevGuideCodingBasicsClient.html)
/DevGuideMvpActivitiesAndPlaces.html)

% Einleitung
% ==========
\section{Einleitung}

Web bietet längst nicht nur Webseiten, sondern vielmehr komplexe Web-Applikationen. Entwicklung von Web-Apps aufwändig, tangiert viele Technologien: HTML/XML. CSS und Javascript/Ajax (Frontend) und beliebige Backends. Nachteil Javascript: keine statische Typkontrolle.

Ansatz GWT: siehe Stichpunkte

\begin{itemize}
\item Anwendungsgebiete: komplexe browserbasierte Client-Server-Anwendungen.
\item Front-End: optimiertes Javascript/AJAX wird aus browserunabhängigen, statisch getypten Java-Code generiert \cite{ref:learnmore}. Vorteil: Nutzung vieler Java-Bibliotheken und -Frameworks problemlos möglich
\item Back-End: kann, muss aber nicht in Java geschrieben werden (gängige Protokolle werden unterstützt)!
\end{itemize}

% Konzepte und Technologien
% =========================
\section{Konzepte und Technologien}
\begin{itemize}
\item Client-Server-Kommunikation (JSON, XML, RPC \cite{ref:learnmore})
\item Client: HTML, CSS, Javascript (Ajax)
\item Model-View-Presenter (MVP)
\item Activities & Places (http://code.google.com/webtoolkit/doc/latest)
\end{itemize}

% Architektur einer GWT-Applikation
% =================================
\section{Architektur einer GWT-Applikation}

\subsection{Vorstellung der Beispielanwendung "`OpenNoteKeeper"'}

\subsection{Projektorganisation einer GWT-Anwendung}
\url{http://code.google.com/webtoolkit/doc/latest/DevGuideOrganizingProjects.html}

\subsection{Model-View-Presenter}
Realisierung des MVC-Patterns in GWT:
EntryPoint, EventBus, Activities&Places, AppController mit eingebautem Historymanagement

\subsection{Erstellung der Benutzerschnittstelle}
Ggf. ergänzende Frameworks nutzen: Smart GWT (Widget-Bibliothek)

\subsection{Serverseitige Funktionserweiterung}

% TODO: ggf streichen und mit in der Architektur als Möglichkeit erwähnen
% Über dem Tellerrand: Ergänzende Frameworks
% ==========================================   
\section{Über dem Tellerrand: Ergänzende Frameworks}

% Zusammenfassung
% ===============
\section{Zusammenfassung}

\end{document}



